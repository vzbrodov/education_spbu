\documentclass[12pt]{article}
\usepackage[margin=1in]{geometry}
\usepackage[english,russian]{babel}
\usepackage{amsmath,amsthm,amssymb,color,latexsym}
\usepackage{geometry}        
\geometry{letterpaper}    
\usepackage{graphicx}

\newtheorem{problem}{Вопрос}

\newenvironment{solution}[1][\it{Ответ}]{\textbf{#1. } }{$\square$}


\begin{document}
	\noindent Экзамен по астрометрии.\hfill \today \\
	Збродов Владислав 22.С03
	
	\hrulefill
	

\section{RS и RF.}
	\begin{problem}
		Что такое RS?
		
		
	\end{problem}

	\begin{solution}
	Reference System (RS), Пространственно-временная система координат (ПВСК) — теоретическая
	концепция пространственно-временных координат, моделей и стандартов, которая позволяет измерять положения и движения объектов в пространстве и времени.
	Для задания ПВСК необходимы:
	\begin{enumerate}
	\item Теория пространства-времени (классическая теория или ОТО)
	\item Система координат и аппарат преобразования координат
	 \item Модели физических явлений, влияющих на измерения
	\item Значения астрономических постоянных (параметры моделей)
			\end{enumerate}
	\end{solution} 
	
		\begin{problem}
		Что такое ICRS?
	\end{problem}
	\begin{solution}
		International Celestial Reference System — международная небесная система координат. Центр — барицентр Солнечной системы. Оси зафиксированы в пространстве относительно внегалактических радиоисточников (направление выбрано по осям в FK5). Основная плоскость — средний экватор FK5, ось Х направлена в точку весеннего равноденствия на эпоху J2000. Шкала времени — TCB (Barycentric Coordinate Time).
	\end{solution}
	
	\begin{problem}
		Что такое ITRS?
	\end{problem}
	
	\begin{solution}
		International Terrestrial Reference System — международная земная система координат. Центр — в геоцентре Земли (включая океан и атмосферу). Ориентация осей определяется из наблюдений IERS (international Earth Rotation and Reference Systems Service). Ось z является средней осью вращения Земли и направлена в опорный полюс (IRP — IERS Reference Pole). Ось x лежит в плоскости опорного меридиана (IRM — IERS Reference	Meridian). Единицей длины является метр, шкалой времени — шкала TCG (Geocentric	Coordinate Time (англ.) — геоцентрическое координационное время). Вращается вместе с
		Землей.
	\end{solution}
	
	\begin{problem}
	Чем реализована ICRS в оптическом диапазоне?
	\end{problem}
	\begin{solution}
Каталоги HIPPARCOS (заменил HCRF), TYCHO, GAIA.
	\end{solution}
	
	
	\begin{problem}
		Чем реализована ICRS в радио-диапазоне?
	\end{problem}
	\begin{solution}
		ICRF3 (2018)
		\begin{itemize}
						\item 4536 радиоисточников с точностью до 0.03 mas.
			\item 303 defining радиоисточника.
			\item Изменен подход к выбору defining источников, предприняты меры по их равномерному распределению.
		\end{itemize}
	\end{solution}
	\begin{problem}
		Что такое RF?
	\end{problem}
	
	\begin{solution}
	Reference Frame, Пространственно-временная система отсчета (ПВСО) — практическая
	реализация концепции ПВСК, созданная с помощью создания шкал времени и каталогов
	опорных источников с известными положениями и собственными движениями.
	\end{solution}
	
	
		\begin{problem}
		Что такое ICRF?
	\end{problem}
	
	\begin{solution}
		International Celestial Reference Frame — практическая реализация ICRS в радиодиапазоне, с центром в барицентре Солнечной Системы, оси фиксированы по 212 (с точностью до 0,5 мсд) и по 608 (с точностью до 1 мсд) радиоисточникам. Главная плоскость совпадает со средним экватором FK5 (в пределах его точности), ось Х направлена на точку весеннего равноденствия $\gamma$ (настоящую), реализуется с помощью РСДБ.
	\end{solution}
		\begin{problem}
		Что такое ITRF?
	\end{problem}
	
	\begin{solution}
		International Thrrestrial Reference Frame — практическая реализация ITRS, с центром с барицентре Земли, координаты фиксированы по (примерно) 800 опорным пунктам	на поверхности Земли, имеющим декартовы координаты $x$, $y$, $z$, $v_x$, $v_y$,$v_z$, ось расположена
		в плоскости Главного (Гринвечевского) меридиана.
	\end{solution}
		\begin{problem}
		С помощью каких наблюдательных средств определяется связь ICRF и ITRF?
	\end{problem}
	
	\begin{solution}
		\begin{itemize}
			\item РСДБ — самая высокая точность для наземных наблюдений — порядка 0,1 мсд.
			\item ГНСС (GPS,ГЛОНАСС) — для определения координат самих пунктов наблюдений.
		\end{itemize}
	\end{solution}
	
	\begin{problem}
	Входит ли плоскость экватора в число базовых плоскостей ICRS?
\end{problem}

\begin{solution}
	Нет. В ICRS оси фиксированы на эпоху J2000, так что плоскость экватора уже не входит
	в число базовых плоскостей.
\end{solution}
	
	
\section{Определения координат небесных тел.}
	
	\begin{problem}
		Что такое абсолютные определения координат небесных тел?
	\end{problem}
	
	\begin{solution}
		Абсолютный способ определения координат заключается в нахождении координат по непосредственному показанию приборов и последующей редукции данных (определения погрешностей), без использования опорных объектов.
	\end{solution}
	
	\begin{problem}
		Что такое относительные определения координат небесных тел?
	\end{problem}
	
	\begin{solution}
		Относительный способ определения координат заключается в нахождении координат с
		помощью уже известных координат ранее наблюдавшихся объектов. При этом в ошибку
		включается ошибка определения координат опорных объектов.
		
		Сводится к измерению разностей координат определяемых и опорных звезд. Использована идея RF.
	\end{solution}
	
	
	
	\begin{problem}
		Написать матрицу преобразования:
	\end{problem}
	
	\begin{solution}
		
		\begin{itemize}
			\item прямоугольных координат (x, y, z) при повороте в положительном направлении (против часовой стрелки, если смотреть с конца оси) системы координат вокруг оси x, y, z	
			на угол $\alpha$, $\beta$, $\gamma$.
		\end{itemize}
		
			\[
			R_{x}(\alpha) =
			\left[ {\begin{array}{ccc}
				1	& 0 & 0 \\
				0	& 	\cos(\alpha) & \sin(\alpha) \\
				0	&  -\sin(\alpha) &	\cos(\alpha)\\	
			\end{array} } \right]
			\]
			\[
			R_{y}(\beta) =
			\left[ {\begin{array}{ccc}
						\cos(\beta)	& 0 &- \sin(\beta) \\
					0	& 	1 & 0  \\
					\sin(\beta)	&  0 &		\cos(\beta)\\	
			\end{array} } \right]
			\]
			\[
			R_{z}(\gamma) =
			\left[ {\begin{array}{ccc}
						\cos(\gamma)	& 	\sin(\gamma) & 0 \\
					-\sin(\gamma)	& 	\cos(\gamma) & 0 \\
					0	&0&1 \\	
			\end{array} } \right]
			\]
		
	\end{solution}
	
	
	\begin{problem}
		Какова точность привязки каталога HIPPARCOS к системе ICRS?
	\end{problem}
	
	\begin{solution}
		Точность привязки осей $\sigma_{\epsilon}$ — $\pm$ 0.6 мсд., Годовое изменение $\omega$ — $\pm$ 0.25 мсд/год.
	\end{solution}
	
	\begin{problem}
		Для чего использовалась главная решетка (Main Grid) в проекте HIPPARCOS?
	\end{problem}
	
	\begin{solution}
		Она увеличивала количество наблюдений одной звезды, таким образом повышая точность
		наблюдений.	
		В решётке было порядка 1000 штрихов, получали некоторый периодический сигнал, который потом обрабатывался с помощью Фурье. Таким образом за один “проход” звезды по
		решётке получались очень точные измерения
		
		(наклон решетки позволяет определять координаты звезды в поле зрения)
	\end{solution}
	
	
	
\section{Задержки}
		\begin{problem}
			Основное уравнение РСДБ:
		\end{problem}
		
		\begin{solution}\\
			$$c\tau = e\cos\delta \cos h + p \sin\delta$$ $$ e = b \cos\psi $$ $$ p = b \sin\psi $$
			
			
			\begin{itemize}
				\item $c -$ скорость света
				\item $\tau -$ задержка РСДБ
				\item $e -$ экваториальная проекция базы
				\item $p -$ полярная проекция базы
				\item $h -$ часовой угол от меридиана базы
				\item $b -$ априорный вектор базы
			\end{itemize}
		\end{solution}
		
		
		\begin{problem}
			Геометрическая задержка
		\end{problem}
		
		\begin{solution}
			Геометрическая задержка связана с базой и скоростью света $$(b,\rho) = c\tau_g$$
			$\rho$ -- направление на радиоисточник
		\end{solution}
		
		
		\begin{problem}
			Групповая задержка.
		\end{problem}
		
		\begin{solution}
			Групповая задержка связана с обработкой сигнала, самим сигналом $$\tau_{gr} = \frac{d\phi}{d\omega}$$
			$\phi$ -- фаза кросскорреляционного сигнала (получена после корреляционной обработки сигналов с обоих телескопов), $\omega$ — циклическая частота сигнала.
		\end{solution}
		
		\begin{problem}
			Гравитационная задержка.
		\end{problem}
		
		\begin{solution}
		Для Земли гравитационная задержка составляет 20 пкс.
		\end{solution}
		
		
		
		\begin{problem}
			Новые астрометрические методы.
		\end{problem}
		
		\begin{solution}
			\begin{center}
				\begin{tabular}{||p{8cm} p{8cm}||} 
					\hline
					Методы(инструменты) & Роль, достижения, результаты  \\ [0.5ex] 
					\hline\hline
					Спутниковые радионавигационные системы и ГНСС & Геоданные и изучение Земли, в т.ч. движение и структура континентальных плит (точность -- метры), изучение состава атмосферы, а также оценка вероятности землетрясений. \\ 
					\hline
					Радиоинтероферометрия со сверхдлинной базой & ICRF и все пять параметров ориентации Земли. \\
					\hline
					Дальномерные измерения (Лазерная локация и радиолокация) & Параметры спутников и грав. поля Земли, координаты полюса Земли. \\
					\hline
					Космическая гравиметрия (исследование гравитационного поля Земли с помощью ИСЗ) & Изучение гравитационного поля Земли.  \\
					\hline
					Космическая астрометрия (преимущественно в оптическом диапазоне) & Возможно, в конечном счете даст новую глобальную систему отсчета.  \\ [1ex] 
					\hline
				\end{tabular}
			\end{center}
		\end{solution}
		
		
		\begin{problem}
			Параметры ориентации Земли и методы их определения:
		\end{problem}
		
		\begin{solution}
			\begin{itemize}
				\item $x_p$, $y_p$ -- координаты полюса, фотозенитная труба, ГНСС, РСДБ, лазерная локация;
				\item $UT1-UTC$ -- угол собственного вращения, пассажный инструмент, часы, ГНСС,
				РСДБ;
				\item $LOD = (UT1 - UTC)$ -- избыточная продолжительность суток;
				\item $\Delta\epsilon$ -- угол прецессии, РСДБ;
				\item $\Delta\psi$ -- угол нутации, РСДБ;
			\end{itemize}
		\end{solution}
		\begin{problem}
			Периоды Эйлера, Чандлера
		\end{problem}
		
		\begin{solution}
			
			Период Эйлера — период свободной нутации — равен 305 дней, найден теоретически, из
			предположения, что Земля — абсолютно твёрдое тело Формула Коткинского:
			$$\phi - \phi_{cp}=x_p \cos\lambda + y_p\sin\lambda$$
			\begin{itemize}
				\item $\phi$ -- широта пункта наблюдения;
				\item $\phi_{cp}$ -- средняя широта пункта;
				\item $x_p$, $y_p$ -- координаты полюса;
				\item $\lambda$ -- долгота пункта.
			\end{itemize}
			
			В действительности был найден другой период (Чандлером), равный 430 дней. Одно из
			первых предположений-объяснений, объясняющих разность периодов, состоит в том, что
			Земля не является абсолютно твердым телом.
		\end{solution}
		
\section{Остальное}
		\begin{problem}
			Основное уравнение космической геодезии:
		\end{problem}
		
		\begin{solution}
			$$r = R + \rho$$
			\begin{itemize}
				\item $r$ -- радиус-вектор ИСЗ в геоцентрической сист.коорд.;
				\item $\rho$ -- радиус-вектор от точки на Земле до спутника (топоцентрическая сист.коорд.);
				\item $R$ -- радиус-вектор этой точки в геоцентрической системе координат.
			\end{itemize}
		\end{solution}
		
		\begin{problem}
			Параллакс $0.01 \pm 0.001$ найти расст. в парсеках оценить среднеквадратичную ошибку:
		\end{problem}
		
		\begin{solution}
			$$r = \frac{1}{\pi^{''}} = 100 pc \pm 10 ( 10\%)$$
		\end{solution}
		
		
		\begin{problem}
			Что такое редукционное уравнение?
		\end{problem}
		
		\begin{solution}
			$$\rho^{'}= M\rho$$ 
Уравнение,исправляющее отличие
			реального инструмента от идеального,т.е.ошибки за
			наклон, азимут и коллимацию.
		\end{solution}
		
		\begin{problem}
			На каких инструментах можно
			делать относительные определения
			координат небесных тел? 
		\end{problem}
		
		\begin{solution}
			вертикальный круг, астрограф,
			мерид.круг
		\end{solution}
		
		\begin{problem}
			Что такое «модель 6
			параметров»?
		\end{problem}
		
		\begin{solution}
			$\zeta=ax+by+c$, $\eta=Ax+By+C$
			– связь рабочих корд. С тангенц. корд.
		\end{solution}
		
		\begin{problem}
			Что такое уравнение яркости?
		\end{problem}
		
		\begin{solution}
			связь
			почернения негатива фотопластинки с
			реальным блеском звезды.
			
			(на рефракторах из-за дисперсии получается "микро-
			спектр" звезды на пластинке):
			$$\zeta = \Sigma ax^iy^jm^kc^l$$
			$$\eta = \Sigma Ax^iy^jm^kc^l$$
		\end{solution}
		
		\begin{problem}
			Чем отличаются шкалы времени UT1
			и UT2?
		\end{problem}
		
		\begin{solution}
			UT2=UT1+$\Delta$Ts, $\Delta T_s$-поправка на сезонные
вариации.
		\end{solution}
		
		\begin{problem}
			Для чего в шкалу UTC вводят
			дополнительную секунду?
		\end{problem}
		
		\begin{solution}
			для
			компенсации замедления Земли
		\end{solution}
		
		\begin{problem}
			 Какая точность определения
			координат небесных объектов
			достигнута методами космической
			астрометрии?
		\end{problem}
		
		\begin{solution}
			1 мсд
		\end{solution}
		
		\begin{problem}
			Какой вклад внес космический
			аппарат им.Хаббла в программу
			HIPPARCOS? 
		\end{problem}
		
		\begin{solution}
			Является одним из методов
			привязки каталога HIPP.к системе ICRF
		\end{solution}
		
		\begin{problem}
			Как определить среднее собственное
			движение звезды по координатам $\mu_\alpha$ и $\mu_\delta$?
		\end{problem}
		
		\begin{solution}
			$$\mu = \sqrt{ (\mu_\alpha\cos\delta)^2+\mu_\delta^2}$$
		\end{solution}
		
		\begin{problem}
			 Для чего использовался картограф в
			программе HIPPARCOS? 
		\end{problem}
		
		\begin{solution}
			Для
			определения ориентации спутника в пр-ве
			и выполнения прогр. TYCHO
		\end{solution}
		
		\begin{problem}
			Угловое разрешение РСДБ при длине
			базы 10000 км достигает величины,
			равной 0,001 секунд души. Как изменится
			угловое разрешение РСБД, если одну
			антенну поместить на Земле, а вторую –
			на Луне
			(длина волны регистрируемого излучения
			остается прежней)?
		\end{problem}
		
		\begin{solution}
			$$\theta= \theta_1\cdot D_1/D_2=0,000025$$
		\end{solution}
		
		\begin{problem}
			Чем определяется начало отсчета
			прямых восхождений в ICRF?
		\end{problem}
		
		\begin{solution}
			Точка весеннего равноденствия
		\end{solution}
		
		\begin{problem}
			Приведите примеры редукционных
			уравнений в астрометрии.
		\end{problem}
		
		\begin{solution}
			Уравнение яркости и Модель в постоянных (модель Тернера)		
		\end{solution}
		
		\begin{problem}
			Какова причина появления
			систематических ошибок в
			астрометрических
			наблюдениях?
		\end{problem}
		
		\begin{solution}
			
			\begin{itemize}
				\item Инструментальные ошибки
				\item  Неполный учет факторов воздействия окружающей среды
				\item  Метод обработки наблюдений
			\end{itemize}		
		\end{solution}
		
		\begin{problem}
			На каких инструментах можно
			определить прямые восхождения
			небесных тел? 
		\end{problem}
		
		\begin{solution}
			пассажный инструмент, меридианный
			круг, астрограф, РСДБ, спутник
			HIPPARCOS		
		\end{solution}
		
		\begin{problem}
			На каких инструментах можно
			делать абсолютные определения
			координат небесных тел? 
		\end{problem}
		
		\begin{solution}
			вертикальный круг,
			мер.круг, РСДБ		
		\end{solution}
		
		\begin{problem}
			Что такое «рабочие координаты»
			небесных объектов?
		\end{problem}
		
		\begin{solution}
			определяют
			звезду в системе фотопластинки		
		\end{solution}
		
		\begin{problem}
			Чем отличается шкала UT1 от
			UT0?
		\end{problem}
		
		\begin{solution}
			$UT1=UT0+\Delta\lambda$, $\Delta\lambda$ --поправка из-за движения полюсов.
		\end{solution}
		
		\begin{problem}
			Какова характерная точность
			определения координат небесных
			объектов методом РСБД?
		\end{problem}
		
		\begin{solution}
			1 мсд	
		\end{solution}
		
		\begin{problem}
			Что такое RGC в программе
			HIPPARCOS?
		\end{problem}
		
		\begin{solution}
			фиксированный
			большой круг, к которому
			редуцируются результаты измерений
			на IGC, полученные в течение одного поворота
			спутника вокруг Земли (10 часов 40 минут)
		\end{solution}
		
		\begin{problem}
			В какой астрометрической
			системе построен каталог
			HIPPARCOS? 
		\end{problem}
		
		\begin{solution}
			HCRF (HIPPARCOS Celestial Reference Frame)
			\begin{itemize}
			\item Основана на каталоге HIPPARCOS (точность $\approx$ 1 mas, как и у ICRF1).
			\item Точность привязки осей 0.6 mas
			\item Годовое изменение 0.25 mas/год.
			\end{itemize}		
		\end{solution}
		
		\begin{problem}
			Почему различаются
			периоды Эйлера и Чандлера
			свободной нутации оси
			вращения Земли?
		\end{problem}
		
		\begin{solution}
			Период
			Эйлера был вычислен
			предполагая, что Земля-абс. ТВ.
			Тело, а пер.Чандлера – упруга	
		\end{solution}
		
		\begin{problem}
			Что такое шкала времени
			UTC?
		\end{problem}
		
		\begin{solution}
			всемирное координированное время. Атомное время, аппроксимирующее UT1: $$|UTC - UT1| \le 0^s .9 $$
			Фактически, $UTC = TAI + T$, где T - секунды координации. Добавляются (и вычитаются) при накоплении
			большего расхождения (больше 0s .9).		
		\end{solution}
		
		\begin{problem}
			Что такое шкала времени
			TDT(TT)?
		\end{problem}
		
		\begin{solution}
		Земное динамическое время. Добавка к TAI для сохранения эфемеридного времени.
		$$TDT=TAI+32,184 sec$$	
		\end{solution}
		
		
\end{document}
